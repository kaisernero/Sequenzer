\subsection{Display (\emph{lcd})}
Die Display Bibliothek steuert das LC-Display und stellt verschiedene Methoden zur Beschreibung und Steuerung dessen zur Verfügung.
Außerdem wurden besondere Methoden für diese Projekt implementiert, die die Anzeige an vorher festgelegten Stellen mit bestimmten Werten beschreibt: \source{write\_pitch}, \source{write\_tone\_length} und \source{write\_tempo}.
\newline
Alle Funktionen des Display benötigen (relativ) viel Zeit, um mit dem Display-Controller zu kommunizieren und sollten demnach nicht in Interrupt Service Routinen benutzt werden.
\subsubsection{\source{void lcd\_init()}}
Das LC-Display wird gestartet und initialisiert über die Funktion \source{void lcd\_init()}. Es wird dann die Initialisierungssequenz für das Display gesendet und es für die weitere Verwendung konfiguriert. Es wird das Display geleert und sowohl der Cursor als auch Blinken ausgeschaltet.
\subsubsection{\source{void lcd\_clear()}}
Eine Funktion, die alles Angezeigte vom Display löscht und den Cursor zurück in die obere, linke Ecke (0, 0) setzt. Sollte nicht zu häufig verwendet werden, da es sonst zu störendem Flackern der Anzeige kommt.
\subsubsection{Schreibfunktionen}
Als grundlegenden Schreibfunktionen, um das Display zu verwenden, sind
\begin{itemize}
    \item \source{void lcd\_write(unsigned char character)},
    \item \source{void lcd\_write\_string(unsigned char string[])} und
    \item \source{void lcd\_write\_int(unsigned int number, int digits)}
\end{itemize}
für die verschiedenen wichtigen Datentypen implementiert. Es können einzelne Zeichen (\source{unsigned character}) gesendet werden, sowie ganze strings, also Felder von Zeichen, die jedoch mit dem terminierendem Zeichen \source{0x0} oder \source{'\textbackslash0'} abgeschlossen werden müssen. Dies passiert automatisch, wenn ein string mit doppelten Anführungszeichen angegeben wird. Um Zahlen (\source{int}) anzuzeigen muss noch ein zweites Argument übergeben werden, das die Anzahl der Stellen bestimmt. Falls die Zahl weniger Stellen besitzt, wird der Rest mit Nullen aufgefüllt, also wird beispielsweise bei der Zahl 45 und drei Stellen \lcdtext{045} angezeigt.
\newline
All diese Funktionen schreiben immer ab der aktuellen Position des Cursors und der Cursor ist nach dem Schreiben beim nachfolgenden Zeichen des letzten neu auf das Display geschriebenen Zeichens.

\subsubsection{\source{void lcd\_set\_cursor(unsigned char x, unsigned char y)}}
Der Cursor kann mithilfe der Funktion \source{void lcd\_set\_cursor(unsigned char x, unsigned char y)} an eine beliebige Stelle im Display gesetzt werden. \source{x} steht für die Position innerhalb der Zeile (beginnend mit 0, also maximal 15) und \source{y} für eine der beiden Zeilen (0 oder 1).

\subsubsection{\source{void write\_pitch(unsigned int pitch)}}
Diese Funktion wandelt die als Zahl angegebene Tonhöhe (wie in X erläutert) %TODO: ref
in die gewohnte Notation (Vorzeichen, Notenname und Oktavnummer, z.B. \lcdtext{\#G4}) um. Sie wird dann auf dem Display an der festgelegten Stelle (0, 0 - linke obere Ecke) für die Tonhöhe angezeigt.

\subsubsection{\source{void write\_tone\_length(enum Tone\_length tone\_length)}}
Mithilfe dieser Funktion wird auf dem Display ab der festgelegten Stelle 0, 1 ein Balkendiagramm als Darstellung der Tonlänge angezeigt. Übergeben wird die in \source{sequence.h} festgelegte Enumeration dafür und je nach Wert zeigt der Balken keine bis vier Segemente: beispielsweise \lcdtext{[OOO.]} oder \lcdtext{[....]}.

\subsubsection{\source{void write\_tempo(unsigned int tempo)}}
Diese Funktion schreibt das Tempo als Zahl in bpm an die dafür festgelegte Stelle auf dem Display: 9, 0.
