\subsection{\source{main.c}}
Die \source{main.c}-Datei ist der Einstiegspunkt des Programms und enthält die Funktion \source{int main()}. Es werden alle unserer Bibliotheken inkludiert, und diese werden gleich zu Beginn durch ihre Initialisierungsfunktionen initialisiert werden können. Danach werden die interrupts freigegeben und das Programm startet die Hauptschleife und damit den interrupt-basierten Ablauf, wie im Abschnitt Grundstruktur und in der Abbildung \ref{img:grundablauf} erläutert.
\newline


Es werden außer den ISRs noch ähnliche Funktionen definiert, die durch Bibliotheken für bestimmte Ereignisse aufgerufen werden. Dies sind \source{void new\_minute()}, das von der Uhrzeit-Bibliothek aufgerufen wird, sobald eine neue Minute angebrochen ist und die Funktion \source{void alarm\_ring()}, die signalisiert, dass ein Alarm ausgelöst worden ist und nun klingeln soll. Diese Funktionen wurden in die \source{main.c} Datei verlegt, da diese Ereignisse Veränderungen in mehreren Bibliotheken auslösen: \source{void new\_minute()} bedeutet, dass die Alarme überprüft werden müssen und die LED-Matrix aktualisiert werden muss;
\source{void alarm\_ring()} bedeutet, dass das Menü in den Alarm-Modus übergehen muss und die Melodie des entsprechenden Alarms gestartet wird.
